% resume.tex
% vim:set ft=tex spell:

\documentclass[10pt,letterpaper]{article}
\usepackage[letterpaper,margin=0.75in]{geometry}
\usepackage[utf8]{inputenc}
\usepackage{mdwlist}
\usepackage[T1]{fontenc}
\usepackage{textcomp}
\usepackage{tgpagella}
\pagestyle{empty}
\setlength{\tabcolsep}{0em}

% indentsection style, used for sections that aren't already in lists
% that need indentation to the level of all text in the document
\newenvironment{indentsection}[1]%
{\begin{list}{}%
	{\setlength{\leftmargin}{#1}}%
	\item[]%
}
{\end{list}}

% opposite of above; bump a section back toward the left margin
\newenvironment{unindentsection}[1]%
{\begin{list}{}%
	{\setlength{\leftmargin}{-0.5#1}}%
	\item[]%
}
{\end{list}}

% format two pieces of text, one left aligned and one right aligned
\newcommand{\headerrow}[2]
{\begin{tabular*}{\linewidth}{l@{\extracolsep{\fill}}r}
	#1 &
	#2 \\
\end{tabular*}}

% make "C++" look pretty when used in text by touching up the plus signs
\newcommand{\CPP}
{C\nolinebreak[4]\hspace{-.05em}\raisebox{.22ex}{\footnotesize\bf ++}}

% and the actual content starts here
\begin{document}

\begin{center}
{\LARGE \textbf{Documentación Alpha PataOn}}

999 E Wacker Drive\ \ \textbullet
\ \ Oficina\ 1001\ \ \textbullet
\ \ Santiago, 2012
\\
(999) 999-9999\ \ \textbullet
\ \ test@example.com
\end{center}

\hrule
\vspace{-0.4em}
\subsection*{Ideas}

\begin{itemize}
	\parskip=0.1em

	\item
	\headerrow
		{\textbf{AAAAAAAAAAAAAAAAAAAAAA}}
		{\textbf{aaaaaaaaa}}
	\\
	\headerrow
		{\emph{AAAAAAAAAAAAAAAAAAAAAA}}
		{\emph{99999999999}}
	\begin{itemize*}
		\item Lorem ipsum dolor sit amet, consectetuer adipiscing elit, sed
		diam nonummy nibh euismod tincidunt ut laoreet dolore magna aliquam
		erat volutpat.
	\end{itemize*}

\end{itemize}

\subsection*{Eventos}

\begin{itemize}
	\parskip=0.1em

	\item
	\headerrow
		{\textbf{Ideas Generales}}
		{\textbf{Casa Miguel}}
	\\
	\headerrow
		{\emph{Ideas que deven ser colocadas en un diferente espacio}}
		{\emph{11/09/2012}}
	\begin{itemize*}
		\item Los eventos son una actividad grupal, en conjuntos de grupos mixtos, los cuales participan
		por ganar dicho evento.
		\item Existe un organizador del evento y participantes. Todos deben ser elegidos de la manera mas justa
		evitando repetición de cual índole.
		\item El organizador deve ser alguien que tenga una suficiente cantidad de Quests, contactos y nivel.
		Ademas el que creo la Quest deve ser el encargado de supervisar el correcto desarrollo del evento.
		\item Los premios se otorgan al finallizar dicho evento y completando las actividades a realizar.
		\begin{itemize} 
		\item El o los organizadores tendran un premio especifico si es que se completo evento.
		\item Los participantes tandran un premio estandar si es que el evento se completa.
		\item Los ganadores del evento tendran un premio en puntaje y experiencia que puede ser canjeado por articulos en la tienda online, los cuales pueden ser regalados a otros usuarios objetivos.
		\end{itemize}
	\end{itemize*}
\end{itemize}

\subsection*{Nivel aspirante/usuario}

\begin{itemize}
	\parskip=0.1em

	\item
	\headerrow
		{\textbf{Ideas Generales}}
		{\textbf{Casa Jean}}
	\\
	\headerrow
		{\emph{Ideas que deven ser colocadas en un diferente espacio}}
		{\emph{11/25/2012}}
	\begin{itemize*}
		\item Los aspirantes a ser miembros de la red deberán registrarse llenando un a formulario con datos mínimos, vale decir correo y Nick Name. Esta forma de registro les permitir´ visualizar las funcionalidades b´sicas del sitio para que el usuario se “tiente” a llenar completamente sus datos lo que le permitir acceder a más secciones dentro del sitio.
		
		\item Un usuario no podrá acceder a ver los datos completos de otro usuario a menos que se encuentre en el mismo nivel dentro del sitio o en alguna actividad en conjunto a la cual haya sido invitado por otro usuario.
		
		\item El usuario con más nivel tiene la posibilidad de crear actividades e invitar usuarios de más bajo nivel del sexo opuesto, lo que le dará la posibilidad de ganar puntos mediante esta acción.
		\item Los usuarios nuevos solo podrán ver una mini ficha de los usuarios más antiguos, la cual tendrá una foto, nivel basado en las características físicas, psicológicas y sociales tanto a nivel de red como de persona. Con esto se quiere decir que un usuario puede destacar de muchas formas, no solo con su apariencia como se ve en otros sitios que solo se basan en las fotos que muchas veces son total mente falsas, cosa que en el sitio no se permitirá si es que el usuario quiere subir de nivel y participar en actividades y concursos.
		\item El sitio contara con un sistema de juegos interactivos a nivel de usuario el cual evaluara a distintos usuarios y seleccionara a los mejores de distintos ámbitos y los emparejara con otros de similares características para realizar actividades que les brindaran puntos con los cuales pueden acceder a más secciones dentro del sitio o juegos diferentes.
		\item Con los puntos se podrán comprar ítems y objetos reales proporcionados los administradores del sitio o patrocinadores.
		\item El sitio contendrá una sección en la cual se mostraran las actividades en tiempo real e históricas realizadas por los usuarios y los premios que recibieron. Esto motivara a los demás usuarios que no se atreven a participar.
		\item Los usuarios no podrán eliminar su cuenta, solo desactivarla o dejarla invisible de los demás, esto estará estipulado explícitamente en el contrato y en FAQs del sitio final.Provenientes
		\item Los usuarios no podrán contactar a nivel de chat a otro si es que no cumplen ciertos requisitos mínimos como:
	    \begin{itemize} 
		\item Haber registrado completamente sus datos y haberlos comprobado.
		\item Haber participado en un número determinado de actividades con usuarios relacionados.
		\item Que la persona que desea contactar este de acurdo con realizar el contacto.
		\end{itemize}
	\end{itemize*}
\end{itemize}
\subsection*{Ideas}

\begin{itemize}
	\parskip=0.1em

	\item
	\headerrow
		{\textbf{PrimeFaces}}
		{\textbf{Casa Jean}}
	\\
	\headerrow
		{\emph{Ideas Provenientes de la pagina de PrimeFaces}}
		{\emph{25/11/2012}}
	\begin{itemize*}
		\item Lorem ipsum dolor sit amet, consectetuer adipiscing elit, sed
		diam nonummy nibh euismod tincidunt ut laoreet dolore magna aliquam
		erat volutpat.
	\end{itemize*}

\end{itemize}
\end{document}
