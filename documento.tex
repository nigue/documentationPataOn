% resume.tex
% vim:set ft=tex spell:

\documentclass[10pt,letterpaper]{article}
\usepackage[letterpaper,margin=0.75in]{geometry}
\usepackage[utf8]{inputenc}
\usepackage{mdwlist}
\usepackage[T1]{fontenc}
\usepackage{textcomp}
\usepackage{tgpagella}
\pagestyle{empty}
\setlength{\tabcolsep}{0em}

% indentsection style, used for sections that aren't already in lists
% that need indentation to the level of all text in the document
\newenvironment{indentsection}[1]%
{\begin{list}{}%
	{\setlength{\leftmargin}{#1}}%
	\item[]%
}
{\end{list}}

% opposite of above; bump a section back toward the left margin
\newenvironment{unindentsection}[1]%
{\begin{list}{}%
	{\setlength{\leftmargin}{-0.5#1}}%
	\item[]%
}
{\end{list}}

% format two pieces of text, one left aligned and one right aligned
\newcommand{\headerrow}[2]
{\begin{tabular*}{\linewidth}{l@{\extracolsep{\fill}}r}
	#1 &
	#2 \\
\end{tabular*}}

% make "C++" look pretty when used in text by touching up the plus signs
\newcommand{\CPP}
{C\nolinebreak[4]\hspace{-.05em}\raisebox{.22ex}{\footnotesize\bf ++}}

% and the actual content starts here
\begin{document}

\begin{center}
{\LARGE \textbf{Documentación Alpha PataOn}}

999 E Wacker Drive\ \ \textbullet
\ \ Oficina\ 1001\ \ \textbullet
\ \ Santiago, 2012
\\
(999) 999-9999\ \ \textbullet
\ \ test@example.com
\end{center}

\hrule
\vspace{-0.4em}
\subsection*{Ideas}

\begin{itemize}
	\parskip=0.1em

	\item
	\headerrow
		{\textbf{AAAAAAAAAAAAAAAAAAAAAA}}
		{\textbf{aaaaaaaaa}}
	\\
	\headerrow
		{\emph{AAAAAAAAAAAAAAAAAAAAAA}}
		{\emph{99999999999}}
	\begin{itemize*}
		\item Lorem ipsum dolor sit amet, consectetuer adipiscing elit, sed
		diam nonummy nibh euismod tincidunt ut laoreet dolore magna aliquam
		erat volutpat.
	\end{itemize*}

\end{itemize}

\subsection*{Eventos}

\begin{itemize}
	\parskip=0.1em

	\item
	\headerrow
		{\textbf{Ideas Generales}}
		{\textbf{Casa Miguel}}
	\\
	\headerrow
		{\emph{Ideas que deven ser colocadas en un diferente espacio}}
		{\emph{11/09/2012}}
	\begin{itemize*}
		\item Los eventos son una actividad grupal, en conjuntos de grupos mixtos, los cuales participan
		por ganar dicho evento.
		\item Existe un organizador del evento y participantes. Todos deben ser elegidos de la manera mas justa
		evitando repetición de cual índole.
		\item El organizador deve ser alguien que tenga una suficiente cantidad de Quests, contactos y nivel.
		Ademas el que creo la Quest deve ser el encargado de supervisar el correcto desarrollo del evento.
		\item Los premios se otorgan al finallizar dicho evento y completando las actividades a realizar.
		\begin{itemize} 
		\item El o los organizadores tendran un premio especifico si es que se completo evento.
		\item Los participantes tandran un premio estandar si es que el evento se completa.
		\item Los ganadores del evento tendran un premio en puntaje y experiencia que puede ser canjeado por articulos en la tienda online, los cuales pueden ser regalados a otros usuarios objetivos.
		\end{itemize}
		asdf 1234
	\end{itemize*}

\end{itemize}
\end{document}
